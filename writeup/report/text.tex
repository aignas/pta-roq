\documentclass{scrreprt}

\usepackage[utf8]{inputenc}
\usepackage[T1]{fontenc}

\usepackage{amsmath, amssymb}

%\usepackage[
    %backend=biber,
    %style=numeric
%]{biblatex}

%\addbibresource{../papers/pta-references.bib}

\newcommand{\bigo}[1]{\ensuremath{\mathcal{O}\left( #1 \right)}}
\renewcommand{\vec}[1]{\ensuremath{\mathbf{#1}}}
\newcommand{\arr}[1]{\ensuremath{\mathrm{#1}}}

% bra ket notation macros
\newcommand{\bra}[1]{\ensuremath{\left\langle#1\right|}}
\newcommand{\ket}[1]{\ensuremath{\left|#1\right\rangle}}
\newcommand{\braket}[2]{\ensuremath{\left\langle #1 \middle| #2 \right\rangle}}
\newcommand{\matrixel}[3]{\ensuremath{\left\langle #1 \middle| #2 \middle| #3 \right\rangle}}

\author{Ignas Anikevičius}
\title{PTA analysis}

\begin{document}
    \maketitle
    \tableofcontents

    \chapter{Introduction}

    \chapter{Background Theory}

    In this chapter I will go through the context of this research as well as
    derivations and definitions of some formulae and mathematical notation used
    through this report.
    %

    \section{Pulsar Timing and Gravitational Wave Search}

    Pulsars are highly predictable cosmic clocks, some of which rival even atomic
    clocks <citation needed>.
    %
    At the moment there is an ongoing effort to establish collaborations between
    many radio telescopes in order to make observations of pulsars across the
    whole sky <citation needed>.
    %
    The so called International Pulsar Timing Array (IPTA) is precisely that.

    The idea, that pulsars can be used for investigation of gravitational
    waves by investigating correlated fluctuations of time of arrival (TOA)
    measurements is not new <citation needed>.
    %
    However, only recently data analysis has been started as sufficient amount of
    data has been gathered <citation needed>.
    %
    The main interest is in seeing whether the residuals of the timing model
    carry any information of gravitational wave source and how to extract it
    quickly.
    %
    The residual is defined as:
    %
    \begin{equation}
        \delta \vec{t} = t_{TOA} - t_{expected TOA}
    \end{equation}

    The procedure is usually to observe the pulsars for a long time and choose
    only the most precise, so called millisecond pulsars, which have a very low
    and steady period and fit the TOAs to obtain the residuals.
    %
    Then the residuals need to be marginalised over the timing model parameters,
    so that the actual data is independent of the fitting model.
    %
    However, the scope of this project is to develop a new way of processing the
    marginalised residuals and hence we shall not concern ourselves with the
    marginalisation of the fitting parameters at this stage and we will assume,
    that the residuals are known to be exact within some error.

    \section{The theoretical model for the signal}

    % TODO I need to find a way to derive the used formula. Look at the Ellis
    % paper and references, because he has done it well, IIRC...

    \section{Search of single sources}

    When the data is being processed, usually Markov Chain Monte Carlo (MCMC)
    methods are chosen to maximise the Bayesian likelihood function.
    %
    The probability of noise defined as a multivariate Gaussian <citation
    needed>:
    %
    \begin{equation}
        p \left( \vec{n} \right) = \cfrac{1}{\sqrt{\det 2\pi \arr{C}}}
        \exp{\left( - \frac{1}{2} \vec{n}^{T} \arr{C}^{-1} \vec{n} \right) }
    \end{equation}
    %
    If we assume that our residuals consist of a signal and a noise, we can
    rearrange the expression for noise:
    %
    \begin{equation}
        \vec{n} = \delta \vec{t} - \vec{h}
    \end{equation}
    %
    where $\vec{h}$ is the signal template which is being fitted.

    The matrix $\arr{C}$ in the inner product in the exponent is called covariance
    matrix, which contains the information about the correlations between the
    noise series $\vec{n}_i$ for each $i$th pulsar.
    %
    It is symmetric and the form of it can be rationalized as follows:
    %
    \begin{equation}
        C_{ij} = C^{GWB}_{ij} + C^{WN}_{ij} + C^{RN}_{ij} + C^{PLN}_{ij} = \left<
        \vec{n}_i \vec{n}_j \right>
    \end{equation}
    %
    where the angular brackets denote the cross-correlation of the noise time
    series from each pulsar <citation needed>.
    %
    As you can see from the same expression, the matrix has various contributions
    from stochastic processes --- Gravitational Wave Background (GWB), the white
    (WN), red (RN) and power-law noises (PLN) <citation needed>.

    The matrix present in the inner product in the exponent makes them
    particularly expensive to evaluate as the operation scales as $\bigo{n^{3}}$,
    where $n$ is the dimensionality of the vectors.
    %
    Also, because of the fact, that the matrix needs to be inverted before
    carrying out evaluations doesn't help either.

    From now on, we shall use a shorthand notation for such inner products, by
    employing the Dirac's bra-ket notation:
    %
    \begin{equation}
        \braket{\vec{a}}{\vec{b}} = \vec{a}^{T} \arr{C}^{-1} \vec{b} 
    \end{equation}
    %
    Given this notation, we can express the logarithm of posterior probability
    function ($\ln p$) as:
    %
    \begin{equation}
        \ln p = -\cfrac{1}{2} \braket{\delta\vec{t}}{\delta\vec{t}} 
        + \braket{\delta\vec{t}}{\vec{h}} 
        -\cfrac{1}{2} \braket{\vec{h}}{\vec{h}}
    \end{equation}

    During the MCMC algorithm run, we would try to maximize $\ln p$ by varying the
    signal template and since the dimensions of the vectors used in this
    calculation tend to be large, these are computationally very intensive.
    %
    It is only the first term, which can be precomputed before doing the
    calculations, however, it does not improve the computation time drastically.

    % FIXME finish

    \section{Reduced Order Modelling}

    Reduced order modelling is not new and such methods rely on a single
    assumption, that the dimensionality of the problem can be reduced by trying to
    assess the information content.
    %
    Quite recently there was a paper published by <canizares citation needed>
    which investigates such models in the context of LIGO <citation needed>
    detector.
    %
    It relies on three crucial steps before the MCMC method application in the
    maximization of the likelihood.

    % FIXME finish

    \chapter{The Reduced Basis Construction}

    % FIXME finish

    It is important to note, that during this stage we can precompute the norm of
    the signal (i.e. \braket{\vec{h}}{\vec{h}}), which is very useful for
    constructing the reduced basis \emph{and} posterior evaluation.
    %
    This means, that during the MCMC cycles we need to compute only one product,
    which we shall speed up by using the Reduced Order Quadrature (ROQ) rule,
    which is described in the next section.

    \chapter{Reduced Order Quadrature (ROQ) rule construction}

    This is a completely general method, which is described in the <Canizares
    citation needed>, but I will include the discussion of the posterior
    calculation here as well.

    We know, that our signal template can be expressed as a linear combination of
    the precomputed Reduced Basis functions:
    %
    \begin{equation}
        \vec{h} \left( x \right) \approx \sum_{i} a_{i}\vec{e}_{i} \left( x \right)
    \end{equation}
    %
    We also make it exactly match at the empirical interpolation points:
    %
    \begin{equation}
        \vec{h} = \sum_{i} a_{i} \vec{e}_{i} \left( F_{k} \right)
    \end{equation}
    %
    Hence, we can reexpress the above equation as a system of equations in
    matrix notation (using summation convention):
    %
    \begin{equation}
        \vec{h}_{i} = A_{ij} a_{j} \implies a_{i} = A^{-1}_{ij} h_{j} 
    \end{equation} 

    By expanding our residuals in a similar fashion, we can easily compute the
    inner product between the residuals and the signal template (using the
    summation convention):
    %
    \begin{align}
        \delta\vec{t} &\approx b_{i} \vec{e}_{i} 
        \implies
        b_{i} = \braket{\delta \vec{t}}{\vec{e}_{i}} 
        \\
        \braket{\delta\vec{t}}{\vec{h}} &\approx b_{i}
        \braket{\vec{e}_{i}}{\vec{e}_{j}} a_i
        = b_{i} G_{ij} a_{i}
        \\
        &= b_{i} G_{ij} A^{-1}_{jk} h_{k} \equiv \widetilde{\delta t}_i h_{i}
    \end{align}
    %
    Since the $\widetilde{\delta \vec{t}}$ can be precomputed before using the
    MCMC method, this greatly speeds up the posterior evaluations during the MCMC
    cycles.

    %
    Given all the derivations we have done, after constructing the ROQ rule, our
    evaluation of the posterior becomes:
    %
    \begin{equation}
        \ln p = -\cfrac{1}{2} \braket{\delta\vec{t}}{\delta\vec{t}} 
        + \widetilde{\delta\vec{t}}^{T}\vec{h} 
        -\cfrac{1}{2} \braket{\vec{h}}{\vec{h}}
    \end{equation}
    %
    of which only the middle term needs to be recalculated and the dot product for
    2 vectors scales as $\bigo{n^{2}}$.
    %
    Thus, by constructing the basis, we can bring the computational cost a lot.

\end{document}

% vim: tw=82:colorcolumn=83:spell:spelllang=en_gb
