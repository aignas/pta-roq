\documentclass{scrartcl}

\usepackage[utf8]{inputenc}
\usepackage[T1]{fontenc}

\usepackage{amsmath, amssymb}

%\usepackage[
    %backend=biber,
    %style=numeric
%]{biblatex}

%\addbibresource{../papers/pta-references.bib}

\newcommand{\bigo}[1]{\ensuremath{\mathcal{O}\left( #1 \right)}}
\renewcommand{\vec}[1]{\ensuremath{\mathbf{#1}}}
\newcommand{\arr}[1]{\ensuremath{\mathrm{#1}}}

% bra ket notation macros
\newcommand{\bra}[1]{\ensuremath{\left\langle#1\right|}}
\newcommand{\ket}[1]{\ensuremath{\left|#1\right\rangle}}
\newcommand{\braket}[2]{\ensuremath{\left\langle #1 \middle| #2 \right\rangle}}
\newcommand{\norm}[1]{\ensuremath{\left|\left| #1 \right|\right|}}
\newcommand{\matrixel}[3]{\ensuremath{\left\langle #1 \middle| #2 \middle| #3 \right\rangle}}

\author{Ignas Anikevičius}
\title{Efficient template norm calculation}

\begin{document}
    \maketitle

    These are some notes on the Norm of the signal template present in the
    likelihood calculations.

    \begin{equation}
        \ln p = -\cfrac{1}{2} \braket{\delta\vec{t}}{\delta\vec{t}} 
        + \braket{\delta\vec{t}}{\vec{h}} 
        -\cfrac{1}{2} \braket{\vec{h}}{\vec{h}}
    \end{equation}

    The last term can be precalculated when we are searching for the reduced
    bases.
    %
    This is because when we are evaluating how well the RB spans the signal
    template space, we calculate the following:
    %
    \begin{equation}
        \norm{\vec{h} - \sum_{i} c_{i} \vec{e}_{i}}^{2}
    \end{equation}
    %
    which could be expanded as:
    %
    \begin{equation}
        \norm{\vec{h}}^{2}
        - 2 \braket{\vec{h}}{\sum_{i} c_{i} \vec{e}_{i}}
        + \norm{\sum_{i} c_{i} \vec{e}_{i}}^2
        =
        \norm{\vec{h}}^{2}
        - 2 \sum_{i} c_{i} \braket{\vec{h}}{\vec{e}_{i}}
        + \sum_{i,j} \mathcal{G}_{ij} c_{i} c_{j} 
    \end{equation}
    %
    where $\mathcal{G}$ is the grammian matrix and $c_i$ are projection
    coeficients.

    This is important, because we reduce the order of calculations as the first
    term can be precomputed before the parameter search, the inner product of the
    second term is used when calculating the $c_{i}$, which means, that we can
    express this as folows:
    %
    \begin{equation}
        \norm{\vec{h}}^{2}
        - 2 \sum_{i,j} \mathcal{G}^{-1}_{ij} \widetilde{c}_{i} \widetilde{c}_{j}
        + \sum_{i,j,k,l} \mathcal{G}_{ij} \mathcal{G}^{-1}_{jk}  \mathcal{G}^{-1}_{il} \widetilde{c}_{i} \widetilde{c}_{j}
        =
        \norm{\vec{h}}^{2}
        - \sum_{i,j} \mathcal{G}^{-1}_{ij} \widetilde{c}_{i} \widetilde{c}_{j}
    \end{equation}
    %
    where
    \begin{align}
        \widetilde{c}_{i} = \braket{\vec{h}}{\vec{e}_{i}}
    \end{align}

    I am not sure about the last result, but if this is true, then the speedups
    should be immense.

\end{document}

% vim: tw=82:colorcolumn=83:spell:spelllang=en_gb
